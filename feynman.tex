\label{eq:84f}
With the solution given in Eq.~\eqref{eq:39f}
\label{eq:85f}
where $U$ is the unitary operator [see Eq.~\eqref{eq:40f}]
\label{eq:86f}
\label{eq:87f}
Hence, the transformation from HP to SP is unitary. At $t=t_i$, states and operators in the two pictures are the same. We see from Eq.~\eqref{eq:86f} that in the HP state vectors are constant in time, while from Eq.~\eqref{eq:87f} the Heisenberg operators evolve with time. Is convenient to keep the temporal label in the Heisenberg states
Eq.~\eqref{eq:87f} ensures the invariance of matrix elements and commutation relations:
Differentiation of Eq. \eqref{eq:87f} 
\label{eq:88f}
\label{eq:89f}
\label{eq:90f}
Differentiating Eq.~\eqref{eq:90f} gives the differential equation of motion operators in the IP:
we substitute Eq.~\eqref{eq:89f} into the Scr\"odinger Eq.~\eqref{eq:84f}
\label{eq:91f}
where, as in Eq.~\eqref{eq:90f}
  \label{eq:92f}
Eq. \eqref{eq:91f} is a Scr\"odinger-like equation with the time dependent Hamiltonian $H_I(t)$. With the interaction switched off (i.e.  we put $H_I=0$), the state vector is constant in time. The interaction leads to the state $|\Phi(t)\rangle$ changing with time. Given that the system is in a state  $|i\rangle$ at an initial time $t=t_i$, i.e.
\label{eq:93f}
the solution of Eq.~\eqref{eq:91f} with this initial condition gives the state $|\Phi(t)\rangle$ of the system at any other time $t$. It follows from the Hermicity of the operator $H_I(t)$ that the time development of the state $|\Phi(t)\rangle$ according to Eq.~\eqref{eq:91f} is an unitary transformation. Accordingly it preserves the normalization of states
Clearly the formalism which we are here developing is not appropriate for the description of bound states but it is particularly suitable for scattering processes. In a collision processes the state vector $|i\rangle$ will define an initial state, long before the scattering occurs ($t_i=-\infty$), by specifying a definite number of particles, with definite properties and far apart from each other so that they do not interact. (For example $|i\rangle$ would specify a definite number of electrons, and positrons with given momenta and spins). In the scattering process, the particles will come close together, collide (i.e interact) and fly apart gain. Eq.~\eqref{eq:91f} determines the state $|\Phi(t)\rangle$ into which the initial state
In order to calculate the $S$--matrix we must solve Eq.~\eqref{eq:91f} for the initial condition \eqref{eq:93f}. These equations can be combined into the integral equation
\label{eq:94f}
From Eq.~\eqref{eq:94f} we can obtain $|\Phi(t_1)\rangle$ at next order:
\label{eq:95f}
The next order of Eq.~\eqref{eq:95f} is
  \label{eq:163f}
\label{eq:164f}
Inserting this expression into (\ref{eq:164f}) gives
  \label{eq:97f}
Let us define the one--particle states as in eq.~\eqref{eq:38f}
From eq.~\eqref{eq:79f} 
  \label{eq:77f}
As established in Sec.~\ref{sec:fock-space-real}, it is convenient to work in the discrete limit where \eqref{eq:26f}
Using the Fourier decomposition  of the scalar field in eq.~\eqref{eq:37f}, and taking into account that 
\label{eq:98f}
By usinbg the commutation relations in eq.~\eqref{eq:32f} we have
  \label{eq:99f}
  \label{eq:100f}
In the lowest order the only term which contributes to the matrix element is the term shown in Eq.~\eqref{eq:97f}
The matrix element at first order in Eq.~\eqref{eq:96f}, between the initial and the final state is then
Using Eqs.~\eqref{eq:99f}\eqref{eq:100f}, we obtain
Comparing with Eq.~\eqref{eq:46f} we have therefore that the relativistic matrix element is
  \label{eq:96f}
\label{eq:156f}
From the Fourier expansions in eqs.~\eqref{eq:83f}, \eqref{eq:78f} we have that $a_s^\dagger$ and $a_s$ are the creation and annihilation operators for particles. As we have only particles (and not antiparticles) in the initial and final states, the only non-zero contribution of the ordered product in eq.~\eqref{eq:156f} must have the order $a^\dagger a^\dagger a\, a$. As $\psi_+$ and $\overline{\psi}_-$ are associated to $a$ and $a^\dagger$ respectively, the only non-zero contribution from the ordered fermion product is
  \label{eq:101f}
We now must sqaure the scattering amplitude, $\mathcal{M}$, and summing up over final spin states, and averaging over the intial spin states, as we did in Eq.~\eqref{eq:81f}. The result that will be obtained in detail in Chapter~\ref{cha:three-body-decays} for the muon--decay is
  \label{eq:102f}
From Eq.~\eqref{eq:74f}
  \label{eq:103f}
The center of mass (CM) frame is defined by the condition in Eq.~\eqref{eq:55f}:
The $\delta$--function in Eq.~\eqref{eq:101f}
  \label{eq:157f}
\label{eq:104f}
  \label{eq:105f}
We already have the expression for $|\mathbf{p}_1|$ as given in eq.~\eqref{eq:71f}. In this case $m_2=0$, and $m_1=m_e$, so that
  \label{eq:158f}
From \eqref{eq:105f}
  \label{eq:106f}
Then, by using Eqs.~\eqref{eq:157f}, \eqref{eq:105f} and \eqref{eq:158f}, and~\eqref{eq:106f},  we have
Replacing back in Eq.~\eqref{eq:102f} and then in Eq.~\eqref{eq:103f} we have
